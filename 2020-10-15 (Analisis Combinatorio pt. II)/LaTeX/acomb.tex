\section{Cadenas}

\paragraph{Para los siguientes ejercicios, considere las letras \textit{ABCDEF}, para cadenas sin repetición.}

\subsection{Ejemplo 1}

Cadenas de longitud 6 que contengan la subcadena \textit{AB}.

\textit{Solución directa}

\textbf{Palomar}
% -----------------
\begin{equation*}
	\palomar AB \palomar
\end{equation*}

\textbf{Fórmula}

\begin{enumerate}
\item Primero escogemos $A$ y $B$ y las ubicamos en un palomar.
$$\left[\binom{2}{2}\cdot\binom{1}{1}\right]$$
\item Escogemos las letras restantes ($C$, $D$, $E$ y $F)$ y las distribuimos en los palomares. Adicionalmente, debemos permutarlas.
% --
\begin{align*}
	\left[\underbset{\text{Seleccionar}}{\binom{6-2}{6-2}}\cdot\underbset{\text{Distribuir}}{\binom{2+(6-2)-1}{6-2}}\cdot\underbset{\text{Permutar}}{\binom{6-2}{1,1,1,1}}\right]
\end{align*}
\item Finalmente, juntamos el resultado de cada paso.
\begin{equation*}
\fn=\left[\binom{2}{2}\cdot\binom{1}{1}\right]
\cdot
\left[\binom{6-2}{6-2}\cdot\binom{2+(6-2)-1}{6-2}\cdot\binom{6-2}{1,1,1,1}\right]
\end{equation*}
\end{enumerate}
% --
\subsection{Ejemplo 2}

Cadenas de longitud 6 que contengan la subcadena \textit{AB}.

\textit{Solución por complemento.}

\textbf{Palomar}
% -----------------
\begin{equation*}
\left\{ \palomar \right\} - [\text{cadenas que NO contengan la subcadena AB}]
\end{equation*}

Esto es:
\begin{equation*}
\left\{ \palomar \right\} - \left(\left\{\palomar B \palomar A \palomar\right\} +\left\{\palomar A \comodin B \palomar\right\} \right)
\end{equation*}


\textbf{Fórmula}

\begin{enumerate}
	\item
	Primero generamos \textbf{todas} las posibles cadenas. Esto corresponde a $\left\{ \palomar \right\}$.
	\begin{align*}
	\left[\binom{6}{6}\cdot\binom{1+6-1}{6}\cdot\binom{6}{1,1,1,1,1,1}\right]
	\end{align*}
	
	\item Lo siguiente es generar todas las cadenas que tienen a la $B$ antes de la $A$.
	Esto corresponde a $\left\{\palomar B \palomar A \palomar\right\}$.
	% --
	\begin{align*}
	\left[\binom{2}{2}\cdot\binom{2}{2}\right]\cdot\left[\binom{6-2}{6-2}\cdot\binom{3+(6-2)-1}{6-2}\cdot\binom{6-2}{1,1,1,1}\right]
	\end{align*}
	
	\item Por último, generamos las cadenas que tienen a la $A$ antes de la $B$, pero sin que estén juntas. Esto corresponde a $\left\{\palomar A \comodin B \palomar\right\}$.
	\begin{align*}
	\left[\binom{2}{2}\cdot\binom{2}{2}\right]\cdot\left[\binom{6-2}{1}\binom{1}{1}\right]\cdot\left[\binom{6-3}{6-3}\cdot\binom{3+(6-3)-1}{3}\cdot\binom{6-3}{1,1,1}\right]
	\end{align*}
	% 
	\begin{align*}
	\fn&=\left[\binom{6}{6}\cdot\binom{1+6-1}{6}\cdot\binom{6}{1,1,1,1,1,1}\right]\\
	&-\left[\binom{2}{2}\cdot\binom{2}{2}\right]\cdot\left[\binom{6-2}{6-2}\cdot\binom{3+(6-2)-1}{6-2}\cdot\binom{6-2}{1,1,1,1}\right]\\
	&-\left[\binom{2}{2}\cdot\binom{2}{2}\right]\cdot\left[\binom{6-2}{1}\binom{1}{1}\right]\cdot\left[\binom{6-3}{6-3}\cdot\binom{3+(6-3)-1}{3}\cdot\binom{6-3}{1,1,1}\right]
	\end{align*}
\end{enumerate}
% --

\subsection{Ejemplo 3}
Cadenas de longitud 6 que contengan las letras $A$ y $B$ juntas.

\paragraph{Palomar}
\textit{Solución directa.}
A diferencia del primer ejemplo, en este caso no importa el orden en que vayan $A$ y $B$, sino que solo importa que estén juntas. Es decir, tanto la subcadena $AB$ como $BA$ son válidas.
% -----------------
\begin{equation*}
\palomar {\color{magenta} AB} \palomar
\end{equation*}

\textbf{Nota:} \textit{El color magenta indica que estas se permutan.}

\paragraph{Fórmula}

\begin{enumerate}
	\item Primero escogemos $A$ y $B$, las ubicamos en un palomar {\color{magenta} y las permutamos}.
	$$\left[\binom{2}{2}\cdot\binom{1}{1}\cdot\binom{2}{1,1}\right]$$
	\item Escogemos las letras restantes ($C$, $D$, $E$ y $F)$ y las distribuimos en los palomares. Adicionalmente, debemos permutarlas.
	% --
	\begin{align*}
	\left[\underbset{\text{Seleccionar}}{\binom{6-2}{6-2}}\cdot\underbset{\text{Distribuir}}{\binom{2+(6-2)-1}{6-2}}\cdot\underbset{\text{Permutar}}{\binom{6-2}{1,1,1,1}}\right]
	\end{align*}
	\item Finalmente, juntamos el resultado de cada paso.
	\begin{equation*}
	\fn=\left[\binom{2}{2}\cdot\binom{1}{1}\cdot\binom{2}{1,1}\right]
	\cdot
	\left[\binom{6-2}{6-2}\cdot\binom{2+(6-2)-1}{6-2}\cdot\binom{6-2}{1,1,1,1}\right]
	\end{equation*}
\end{enumerate}
% ---------
\subsection{Ejemplo 4}
Cadenas de longitud 6 que contengan las letras $A$ y $B$ juntas.

\paragraph{Palomar}
\textit{Solución por complemento.}

Al conjunto de todas las posibles cadenas, le quitaremos las cadenas que tengan la $A$ y la $B$ separadas.
\begin{align*}
\left\{\palomar\right\}-\left\{\text{Cadenas que tengan $A$ y $B$ separadas}\right\}
\end{align*}
Esto es
\begin{align*}
\left\{\palomar\right\}-\left\{\palomar \cmag{A} \comodin \cmag{B} \palomar \right\}
\end{align*}
\paragraph{Fórmula}
\begin{enumerate}
\item Primero generamos \textbf{todas} las posibles cadenas. Esto corresponde a $\left\{ \palomar \right\}$.
\begin{align*}
\left[\binom{6}{6}\cdot\binom{1+6-1}{6}\cdot\binom{6}{1,1,1,1,1,1}\right]
\end{align*}
\item Lo siguiente es contar las cadenas que no tengan $A$ y $B$ juntas.
\begin{align*}
\left[\binom{2}{2}\cdot\binom{2}{2}\cdot \binom{2}{1,1}\right]\cdot\left[\binom{6-2}{1}\cdot\binom{1}{1}\right]\cdot\left[\binom{6-3}{6-3}\cdot\binom{3+(6-3)-1}{6-3}\cdot\binom{6-3}{1,1,1}\right]
\end{align*}
\end{enumerate}

Finalmente:
\begin{align*}
\fn&=\left[\binom{6}{6}\cdot\binom{1+6-1}{6}\cdot\binom{6}{1,1,1,1,1,1}\right]\\
&-\left[\binom{2}{2}\cdot\binom{2}{2}\cdot \binom{2}{1,1}\right]\cdot\left[\binom{6-2}{1}\cdot\binom{1}{1}\right]\cdot\left[\binom{6-3}{6-3}\cdot\binom{3+(6-3)-1}{6-3}\cdot\binom{6-3}{1,1,1}\right]
\end{align*}

\subsection{Ejemplo 5}
Cadenas de longitud 6 tal que la $D$ y la $E$ no estén juntas.


\paragraph{Palomar}
\textit{Solución Directa}

Ubicamos un palomar \textbf{no vacío} entre $D$ y $E$. Además, es importante tener en cuenta que no importa en qué orden estén las dos letras, es decir, puede ir tanto la $D$ antes de la $E$ o la $D$ antes de la $E$, por lo que debemos permutarlas.

% -----------------
\begin{equation*}
\palomar \cmag{D} \comodin  \cmag{E} \palomar
\end{equation*}

\paragraph{Fórmula}
\begin{enumerate}
\item Primero seleccionamos $D$ y $E$ y las permutamos.
\begin{align*}
\left[\binom{2}{2}\cdot\binom{2}{2}\cdot\binom{2}{1,1}\right]
\end{align*}
\item Ahora debemos asegurarnos de que en el comodín vaya al menos una letra. Por lo que escogemos una cualquiera de las letras restantes.
\begin{align*}
\left[\binom{6-2}{1}\cdot\binom{1}{1}\right]
\end{align*}
\item Por último, seleccionamos las ahora restantes y las distribuimos en los tres palomares y las permutamos.
\begin{align*}
\left[\binom{6-3}{6-3}\cdot\binom{3+(6-3)-1}{6-3}\cdot\binom{6-3}{1,1,1}\right]
\end{align*}
\end{enumerate}
Por lo tanto, la respuesta final es
\begin{align*}
\fn=\left[\binom{2}{2}\cdot\binom{2}{2}\cdot\binom{2}{1,1}\right]\cdot\left[\binom{6-2}{1}\cdot\binom{1}{1}\right]\cdot\left[\binom{6-3}{6-3}\cdot\binom{3+(6-3)-1}{6-3}\cdot\binom{6-3}{1,1,1}\right]
\end{align*}

% --
\subsection{Ejemplo 6}
Cadenas de longitud 6 tal que la $D$ y la $E$ no estén juntas.


\paragraph{Palomar}
\textit{Solución por Complemento}

Esta cantidad se puede definir como el total de palabras posibles menos las palabras que tienen a $D$ y $E$ juntas.
\begin{align*}
\left\{\palomar\right\}-\left\{\text{Cadenas que tienen a $D$ y $E$ juntas} \right\}
\end{align*}
Esto es
\begin{align*}
\left\{\palomar\right\}-\left\{\palomar {\color{magenta} DE} \palomar\right\}
\end{align*}

\paragraph{Fórmula}
\begin{enumerate}
\item Lo primero es definir el gran total:
\begin{align*}
\left[\binom{6}{6}\cdot\binom{1+6-1}{6}\cdot\binom{6}{1,1,1,1,1,1}\right]
\end{align*}
\item Recordemos que la cantidad de palabras a partir las letras definidas que se pueden formar de tal modo que $D$ y $E$ estén juntas es (equivalente a cuando $A$ y $B$ están juntas).
\begin{align*}
\left[\binom{2}{2}\cdot\binom{1}{1}\cdot\binom{2}{1,1}\right]
\cdot
\left[\binom{6-2}{6-2}\cdot\binom{2+(6-2)-1}{6-2}\cdot\binom{6-2}{1,1,1,1}\right]
\end{align*}
\end{enumerate}

\begin{align*}
\fn&=\left[\binom{6}{6}\cdot\binom{1+6-1}{6}\cdot\binom{6}{1,1,1,1,1,1}\right]\\&-
\left[\binom{2}{2}\cdot\binom{1}{1}\cdot\binom{2}{1,1}\right]
\cdot
\left[\binom{6-2}{6-2}\cdot\binom{2+(6-2)-1}{6-2}\cdot\binom{6-2}{1,1,1,1}\right]
\end{align*}


