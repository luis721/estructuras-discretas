\section{Funciones Generadoras Exponenciales (EGFs)}

\subsection{Ejemplo 1}

Dada la secuencia $\se{1}{n\geq0,\,n\bmod2=0}$.

Esta secuencia es
$$\langle1,\,0,\,1,\,0,\dots,\,1,\,0,\dots\rangle$$.

Definimos $F(z)$ como la EGF de esta secuencia.
\begin{align*}
F(z)&=\sum_{\underset{n\bmod 2=0}{n\geq0}}{1\cdot\ek{n}}
\end{align*}

Expandiendo esta serie se obtiene
\begin{align*}
F(z)&=1\cdot\ek{0}+1\cdot\ek{2}+1\cdot\ek{4}+\cdots+1\cdot\cfrac{z^{2n}}{(2n)!}+\cdots
\end{align*}

\subsubsection*{Una posible solución}

Podemos definir este $f_n$ como
\begin{align*}
f_n&=f_{n-2},\,n>1,\,f_0=1,\,f_1=0.
\end{align*}
% --
\begin{table}[ht!]	\centering
\begin{tabular}{|c|c|}
\hline
$n$ & $f_n$\\
\hline
0 & 1\\
1 & 0\\
2 & 1\\
3 & 0\\
4 & 1\\
\vdots & \vdots\\
\hline
\end{tabular}
\caption{Primeros valores de $f_n$}
\end{table}

El valor de $f_n$ es
$$f_n=\cfrac{1+(-1)^n}{2}.$$

Para los valores pares, esto es 
$$f_{2n}=\cfrac{1+(-1)^{2n}}{2}=\cfrac{1+1}{2}=1.$$
Para los valores impares, esto es 
$$f_{2n+1}=\cfrac{1+(-1)^{2n+1}}{2}=\cfrac{1-1}{2}=0.$$

Por lo tanto, podemos escribir $F(z)$ de la siguiente forma
\begin{align*}
F(z)&=\sum_{n\geq0}{\cfrac{1+(-1)^n}{2}\cdot\ek{n}}
\end{align*}

A partir de esto:
\begin{align*}
F(z)&=\sum_{n\geq0}{\cfrac{1+(-1)^n}{2}\cdot\ek{n}}=\cfrac{1}{2}\cdot\left(\underbset{e^z}{\sum_{n\geq0}{1\cdot\ek{n}}}+\underbset{e^{-z}}{\sum_{n\geq0}{(-1)^n\cdot\ek{n}}}\right)\\
F(z)&=\cfrac{e^z+e^{-z}}{2}.
\end{align*}

\subsection{Otra posible solución}
\begin{align*}
e^z&\rightarrow \langle+1,+1,+1,+1,+1,+1,\dots,+1,+1\,\dots\rangle\\
e^{-z}
	&\rightarrow \langle+1,-1,+1,-1,+1,-1,\dots,+1,-1\,\dots\rangle\\
e^z+e^{-z}&\rightarrow\langle+2,\cc0,+2,\cc0,+2,\cc0,\dots,+2,\cc0\,\dots\rangle\\
\cfrac{e^z+e^{-z}}{2}&\rightarrow\langle+1,\cc0,+1,\cc0,+1,\cc0,\dots,+1,\cc0\,\dots\rangle\\
\end{align*}

\subsection{Ejemplo 2}

Hallar la EGF de $\se{n}{n\geq0,\,n\bmod2=0}$.

Esta secuencia la podemos reescribir. Y resulta lo siguiente:
$$\se{n\cdot\cfrac{1+(-1)^n}{2}}{n\geq0}$$

\begin{align*}
F(z)&=\sum_{n\geq0}{n\cdot\cfrac{1+(-1)^n}{2}\cdot\ek{n}}&=\cfrac{1}{2}\cdot\left(\underbset{z\cdot e^z}{\sum_{n\geq0}{n\cdot\ek{n}}}+\underbset{A(z)}{\sum_{n\geq0}{n\cdot(-1)^n\cdot\ek{n}}}\right)
\end{align*}

\textbf{Es necesario mostar cómo se halló quién es $A(z).$}
\begin{align*}
A(z)={\sum_{n\geq0}{n\cdot(-1)^n\cdot\ek{n}}}
\end{align*}
Sabemos que 
\begin{align*}
e^{-z}&=\sum_{n\geq0}{(-1)^n\cdot\ek{n}}\\
z\cdot e^{-z}&=\sum_{n\geq0}{n\cdot (-1)^{n-1}\cdot\ek{n}}\\
-z\cdot e^{-z}&=\underbset{A(z)}{\sum_{n\geq0}{n\cdot (-1)^{n}\cdot\ek{n}}}.
\end{align*}

Dado que ya tenemos $A(z)$, entonces
\begin{align*}
F(z)&=\cfrac{z\cdot e^z-z\cdot  e^{-z}}{2}.
\end{align*}

\subsection{Ejemplo 3}

Dada la siguiente EGF
$$F(z)=\cfrac{e^{3z}-e^{2z}}{z},$$ hallar $n![z^n]F(z)$.

\subsubsection*{Solución}
\begin{align*}
F(z)&=\cfrac{e^{3z}-e^{2z}}{z}=\cfrac{e^{2z}\cdot(e^z-1)}{z}\\
F(z)&=e^{2z}\cdot \cfrac{e^z-1}{z}.
\end{align*}
De las EGFs básicas sabemos lo siguiente
\begin{align*}
e^{2z}&=\sum_{n\geq0}{2^n\cdot\ek{n}}\\
\cfrac{e^z-1}{z}&=\sum_{n\geq0}{\cfrac{1}{n+1}\cdot\ek{n}}
\end{align*}

Aplicando la operación de convolución binomial se obtiene:
\begin{align*}
e^{2z}\cdot \cfrac{e^z-1}{z}&=\sum_{n\geq0}{\sum_{0\leq k\leq n}{\binom{n}{k}\cdot 2^k\cdot \cfrac{1}{n-k+1}\cdot\ek{n}}}
\end{align*}

En este caso, dejaremos esta expresión abierta como resultado. Por lo tanto
\begin{align*}
f_n&=\sum_{0\leq k\leq n}{\binom{n}{k}\cdot 2^k\cdot \cfrac{1}{n-k+1}}.
\end{align*}
