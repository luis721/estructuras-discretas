\section{Funciones Generadoras Exponenciales (EGFs)}

Dada una secuencia $\seq{f_n}_{n\geq0}$, entonces su EGF $F(z)$ viene dada de la siguiente manera:

\subsection{Definición}

\begin{align*}
F(z)&=f_0\cdot\ek{0}+f_1\cdot\ek{1}+f_2\cdot\ek{2}+f_3\cdot\ek{3}+\cdots=\sum_{n\geq0}{f_n\cdot \ek{n}}.
\end{align*}

Esto es: Convertimos los términos de las secuencia el coeficientes de una serie de potencias.

Una serie de potencias es algo de la forma
\begin{align*}
F(z)&=\sum_{n\geq0}{f_n\cdot k(z)}
\end{align*}

Donde $k(z)$ es el \textit{kernel}, que debe ser de la forma $z^{p(n)}\cdot q(n)$.

Un ejemplo básico sería:
\begin{align*}
F(z)&=\sum_{n\geq0}{f_n\cdot z^n}
\end{align*}

En el caso de las funciones generadoras exponenciales (EGFs) el \textit{kernel} es
$\frac{z^n}{n!}$.

Lo que significa que la EGF de una secuencia $\seq{f_n}_{n\geq0}$ es:
\begin{align*}
F(z)&=\sum_{n\geq0}{f_n\cdot \ek{n}}
\end{align*}

El \textit{kernel} en una EGF \textbf{debe} estar explícito.

\subsection{Ejemplos}

\subsubsection{Ejemplo 1}

Dada la secuencia $\seq{1}_{n\geq0}$, determinar su EGF. 

Esto significa que $f_n=1,\,n\geq0$. Por lo tanto esto representa la secuencia $\langle1,1,1,1,\dots,1,\dots\rangle$. 
En general $f_n$ es $\langle f_0,f_1,f_2,f_3,\dots,f_n,\dots\rangle$.

Dado esto, la EGF de $\seq{1}_{n\geq0}$ es:
\begin{align*}
F(z)&=\sum_{n\geq0}{\underbset{f_n}{1}\cdot \ek{n}}=1\cdot\ek{0}+1\cdot\ek{1}+1\cdot\ek{2}+1\cdot\ek{3}+\cdots+1\cdot\ek{n}+\cdots\\
F(z)&=e^z.
\end{align*}

Esto se escribe:
$$n![z^n]e^z=1.$$

Esto se lee como \textit{el coeficiente de la EGF $F(z)=e^z$ es 1, para $n\geq0$}.

\subsubsection{Ejemplo 2}

Dada la secuencia $\seq{2}_{n\geq0}$, determinar su EGF.

Esto significa que la secuencia es:
\begin{align*}
\langle2,\,2,\,2,\,2\,,\dots,\,2,\,\dots\rangle
\end{align*}

Es decir, $f_n=2$. 

Primero definimos $F(z)$ como su EGF:

\begin{align*}
F(z)&=\sum_{n\geq0}{2\cdot \ek{n}}
\end{align*}

Este ejercicio se puede resolver de dos maneras distintas

\paragraph{Una posible solución}

\begin{align*}
F(z)&=\sum_{n\geq0}{2\cdot \ek{n}}=2\cdot\ek{0}+2\cdot\ek{1}+\cdots+2\cdot\ek{n}+\cdots
\end{align*}
Entonces despejamos el 2 por ser una constante respecto al índice de la sumatoria ($n$) y por estar multiplicando a todos los términos.

\begin{align*}
F(z)&=2\cdot\underbset{e^z}{\left(\sum_{n\geq0}{1\cdot \ek{n}}\right)}=2\cdot\underbset{e^z}{\left(1\cdot\ek{0}+1\cdot\ek{1}+\cdots+1\cdot\ek{n}+\cdots\right)}\\
F(z)&=2\cdot e^z.
\end{align*}

Es decir: La función generadora de $\seq{2}_{n\geq0}$ es $2\cdot e^z$.

\paragraph{Otra forma de resolverlo}

\begin{align*}
F(z)&=\sum_{n\geq0}{2\cdot \ek{n}}=2\cdot\ek{0}+2\cdot\ek{1}+\cdots+2\cdot\ek{n}+\cdots
\end{align*}

Si en lugar de un $2$ hubiese un 1 en el coeficiente, esto sería más sencillo.
\begin{align*}
F(z)&=\sum_{n\geq0}{(1+1)\cdot \ek{n}}=(1+1)\cdot\ek{0}+(1+1)\cdot\ek{1}+\cdots+(1+1)\cdot\ek{n}+\cdots
\end{align*}

Recordemos que una propiedas de las sumatorias es:
\begin{align*}
\sum_{P(n)}{\left(f_n+g_n\right)}=\sum_{P(n)}{f_n}+\sum_{P(n)}{g_n}.
\end{align*}
Por lo tanto, en nuestro ejemplo tenemos
\begin{align*}
F(z)&=\underbset{e^z}{\left(\sum_{n\geq0}{1\cdot \ek{n}}\right)}+\underbset{e^z}{\left(\sum_{n\geq0}{1\cdot \ek{n}}\right)}\\
&=e^z+e^z=2\cdot e^z\\
F(z)&=2\cdot e^z.
\end{align*}

\paragraph{Comprobación}
Es posible comprobar nuestra solución. Esto se hace de la siguiente manera:

Si $F(z)$ es la EGF de $f_n$, entonces, la $n$-ésima derivada de $F(z)$ evaluada en 0 debe ser igual a $f_n$.

Es decir $F^{(n)}(0)=f_n.$

En este caso, la $n$-ésima derivada de $2\cdot e^z$ evaluada en 0 debe ser igual a 2.

\subsubsection{Ejemplo 3}

Hallar la EGF de $\se{n+1}{n>4}$.

Primero definimos $F(z)$ como la EGF de esta secuencia.

\begin{align}
\label{eq:nmas1}
F(z)&=\sum_{n>4}{\left(n+1\right)}\cdot\ek{n}
\end{align}

\paragraph{Una forma de resolverlo}
En este enfoque partimos de $F(z)$.
Ahora, cambiaremos el dominio de estas sumatorias, de modo que queden como está definido para $F(z)$ en \eqref{eq:nmas1}.

\begin{align*}
\underbset{F(z)}{\sum_{n\geq5}{(n+1)\cdot \ek{n}}}&=\left(\sum_{n\geq5}{n\cdot \ek{n}}\right)+\left(\sum_{n\geq5}{1\cdot \ek{n}}\right)&\langle 0, 0,0,0,0,\underbset{n\geq5}{6,7,8,\,\cdots,n+1,\dots}\rangle
\end{align*}
Esto se puede re-escribir como
\begin{align*}
F(z)&=\left(\underbset{ze^z}{\sum_{n\geq0}{n\cdot \ek{n}}}-\sum_{0\leq n\leq 4}{n\cdot\ek{n}}\right)+\left(\underbset{e^z}{\sum_{1\geq0}{1\cdot \ek{n}}}-\sum_{0\leq n\leq 4}{1\cdot\ek{n}}\right)\\
F(z)&=ze^z+e^z-\sum_{0\leq n\leq 4}{n\cdot\ek{n}}-\sum_{0\leq n\leq 4}{1\cdot\ek{n}}
\end{align*}

Expandiendo las sumatorias se obtiene
\begin{align*}
F(z)=&ze^z+e^z\\&-\left(0\cdot\ek{0}+1\cdot\ek{1}+2\cdot\ek{2}+3\cdot\ek{3}+4\cdot\ek{4}\right)
-\left(1\cdot\ek{0}+1\cdot\ek{1}+1\cdot\ek{2}+1\cdot\ek{3}+1\cdot\ek{4}\right)\\
=&ze^z+e^z\\&-\left(0\cdot\ek{0}+1\cdot\ek{1}+2\cdot\ek{2}+3\cdot\ek{3}+4\cdot\ek{4}
+1\cdot\ek{0}+1\cdot\ek{1}+1\cdot\ek{2}+1\cdot\ek{3}+1\cdot\ek{4}\right)\\
=&ze^z+e^z\\&-\left(1\cdot\ek{1}+2\cdot\ek{2}+3\cdot\ek{3}+4\cdot\ek{4}
+1\cdot\ek{0}+1\cdot\ek{1}+1\cdot\ek{2}+1\cdot\ek{3}+1\cdot\ek{4}\right)\\
=&ze^z+e^z-
\left(1\cdot\ek{0}+2\cdot\ek{1}+3\cdot\ek{2}+4\cdot\ek{3}+5\cdot\ek{4}\right)\\
F(z)=&ze^z+e^z-\left(1+2z+3\cdot\ek{2}+4\cdot\ek{3}+5\cdot\ek{4}\right)
\end{align*}

\paragraph{Otra posible solución}
En esta primera solución a este problema, partiremos de la EGF básica.

\begin{align*}
e^z&=\sum_{n\geq0}{\ek{n}}
\end{align*}

La idea es aplicar distintas operaciones sobre esta EGF hasta llegar a la expresión de \eqref{eq:nmas1}.

Entonces aplicamos la operación de multiplicación por el índice.
% ---
\begin{align*}
z\cdot e^z&=\sum_{n\geq0}{n\cdot \underbset{f_{n-1}}{1}\cdot \ek{n}}&\langle0, 1, 2,\dots,n,\dots\rangle\\
z\cdot e^z&=\sum_{n\geq0}{n\cdot \ek{n}}&\langle0, 1, 2,\dots,n,\dots\rangle
\end{align*}

El siguiente paso es procurar que el dominio de esta sumatoria sea el mismo de \eqref{eq:nmas1}.

\begin{align*}
\sum_{n\geq5}{n\cdot \ek{n}}&=\sum_{n\geq0}{n\cdot\ek{n}}-\sum_{0\leq n\leq 4}{n\cdot \ek{n}}\\
\sum_{n\geq5}{n\cdot \ek{n}}+\sum_{n\geq5}{1\cdot \ek{n}}&=\sum_{n\geq0}{n\cdot\ek{n}}-\sum_{0\leq n\leq 4}{n\cdot \ek{n}}+\sum_{n\geq5}{1\cdot \ek{n}}\\
\underbset{F(z)}{\sum_{n\geq5}{\left(n+1\right)\cdot \ek{n}}}&=\sum_{n\geq0}{n\cdot\ek{n}}-\sum_{0\leq n\leq 4}{n\cdot \ek{n}}+\sum_{n\geq0}{1\cdot \ek{n}}-\sum_{0\leq n\leq 4}{1\cdot \ek{n}}\\
F(z)&=\left(\sum_{n\geq0}{n\cdot\ek{n}}\right)+\left(\sum_{n\geq0}{1\cdot\ek{n}}\right)-\left(\sum_{0\leq n\leq 4}{n\cdot \ek{n}}+\sum_{0\leq n\leq 4}{1\cdot \ek{n}}\right)
\end{align*}
Esta parte ya está resuelta en la forma de resolverlo que se mostró antes.