\section{Funciones Generadoras Exponenciales (EGFs)}


\subsection{Multiplicación por el índice}

\subsubsection{Ejemplo 1}
Dada la EGF $F(z)=e^z$, 

\begin{align*}
e^z&=\sum_{n\geq0}{\underbset{f_n}{1}\cdot\ek{n}}
\end{align*}

el resultado aplicar la multiplicación por el índice a esta EGF es el siguiente:

\begin{align*}
z\cdot e^z&=\sum_{n\geq0}{n\cdot \underbset{f_{n-1}}{1}\cdot\ek{n}}
\end{align*}

\subsubsection{Ejemplo 2}

Dada la EGF $F(z)$ definida como
\begin{align*}
F(z)=\cfrac{e^z-1}{z}&=\sum_{n\geq0}{\underbset{f_n}{\cfrac{1}{n+1}}\cdot \ek{n}}
\end{align*}

Al aplicar la operación de multiplicación por el índice, se obtiene:

\begin{align*}
zF(z)&=z\cdot \cfrac{e^z-1}{z}=\sum_{n\geq0}{n\cdot \underbset{f_{n-1}}{\cfrac{1}{n}}\cdot \ek{n}}\\
&=e^z-1=\sum_{n\geq1}{n\cdot\cfrac{1}{n}\cdot \ek{n}}=\sum_{n\geq1}{1\cdot \ek{n}} & \langle0,1,1,\dots,1,\dots\rangle.
\end{align*}
Si $f_n=\frac{1}{n+1}$, entonces $f_{n-1}=\frac{1}{n}$.

\subsection{Composición de funciones}

Dada la EGF $F(z)=e^z$, 

\begin{align*}
F(z)=e^z&=\sum_{n\geq0}{\ek{n}}\\
F(2z)=e^{2z}&=\sum_{n\geq0}{\cfrac{(2\cdot z)^n}{n!}}=\sum_{n\geq0}{2^n\cdot \ek{n}}\\
\end{align*}
% --
Es decir:
$$n![z^n]e^{2z}=2^n.$$
ó
$$f_n=2^n,\,n\geq0.$$
\subsection{Desplazamiento}

Hallar la EGF de $\se{\binom{n}{3}}{n\geq2}$.

Primero definimos $F(z)$ como la EGF de la secuencia.
\begin{align*}
F(z)&=\sum_{n\geq2}{\left(n-2\right)\cdot\ek{n}}
\end{align*}

Partiremos de la siguiente EGF.
% --
\begin{align*}
ze^z=\sum_{n\geq0}{n\cdot\ek{n}}
\end{align*}
% --
Vamos a hacer una sustitución en la sumatoria. Esta es la de sustituir $n$ por $n-2$.
% --
\begin{align*}
ze^z&=\sum_{n-2\geq0}{(n-2)\cdot\cfrac{z^{n-2}}{(n-2)!}}\\
ze^z&=\sum_{n\geq2}{(n-2)\cdot\cfrac{z^{n-2}}{(n-2)!}}
\end{align*}
% --
Multiplicamos por $z^2$ de ambos lados
\begin{align*}
z^3e^z&=\sum_{n\geq2}{(n-2)\cdot\cfrac{z^n}{(n-2)!}}=\sum_{n\geq2}{(n-2)\cdot\cfrac{z^n}{(n-2)!}\cdot\cfrac{(n-1)\cdot n}{(n-1)\cdot n}}\\
&=\sum_{n\geq2}{(n-2)\cdot(n-1)\cdot n\cdot \ek{n}}\\
z^3e^z&=\sum_{n\geq2}{\binom{n}{3}\cdot 3!\cdot \ek{n}}\\
\end{align*}

\paragraph{Nota:}
\begin{align*}
\binom{n}{3}&=\cfrac{n!}{3!\cdot (n-3)!}=\cfrac{n\cdot (n-1)\cdot (n-2)\cdot (n-3)!}{3!\cdot (n-3)!}\\
&=\cfrac{n\cdot (n-1)\cdot(n-2)}{3!}.
\end{align*}
