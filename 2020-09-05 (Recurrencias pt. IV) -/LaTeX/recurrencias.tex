\section{Relaciones de recurrencia}

\subsection{Ejemplo 1}
Resolver la siguiente relación de recurrencia.

$$f_n=f_{n-1}+(-1)^n,\,f_1=2,\,n>1$$

\subsection{Metodo de iteraciones}
\begin{align*}
k=1.\,f_n&=f_{n-1}+(-1)^n\\
k=2.\,f_n&=f_{n-2}+(-1)^{n-1}+(-1)^n\\
k=3.\,f_n&=f_{n-3}+(-1)^{n-2}+(-1)^{n-1}+(-1)^n
\end{align*}

A partir de esto escribimos la ecuación paramétrica:
\begin{align*}
f(n,k)=f_{n-k}+\sum_{i=0}^{k-1}{(-1)^{n-i}}
\end{align*}

A partir del caso base, tenemos:
$$n-k=1\implies k=n-1\implies f_{n-k}=f_1.$$

Aplicamos esto sobre la ecuación paramétrica y obtenemos entonces:
\begin{align*}
f_n&=f_{1}+\sum_{i=0}^{n-2}{(-1)^{n-i}}.
\end{align*}

Pero sabemos que
\begin{align*}
f_n&=f_{1}+\sum_{i=0}^{n-2}{(-1)^{n-i}}\\
&=f_1+(-1)^{n-0}+(-1)^{n-1}+(-1)^{n-2}+\cdots+{(-1)^{n-(n-2)}}\\
&=f_1+(-1)^n\cdot\left((-1)^0+(-1)^{-1}+(-1)^{-2}+(-1)^{-3}+\cdots+(-1)^{n-2}\right)
\\
&=f_1+\sum_{i=0}^{n-2}{(-1)^{-i}}=f_1+\sum_{i=0}^{n-2}{\cfrac{1}{(-1)^i}}=f_1+\sum_{i=0}^{n-2}{\left(\cfrac{1}{-1}\right)^i}
\end{align*}

El siguiente paso es resolver la siguiente expresión
\begin{align*}
f_n&=f_1+\sum_{i=0}^{n-2}{\left(\cfrac{1}{-1}\right)^i}=f_1+\sum_{i=0}^{n-2}{\left(-1\right)^i}
\end{align*}

Esta sumatoria la resolvemos como la suma de una progresión geométrica.
\begin{align}
\label{eq:sum-geom}
G(n, r, a) = \sum_{i=0}^{n-1}{a\cdot r^i}=a\cdot\cfrac{1-r^n}{1-r}
\end{align}

\begin{align*}
f_n&=f_1+\underbset{G(n-1,-1,1)}{\sum_{i=0}^{n-2}{\left(\cfrac{1}{-1}\right)^i}}=f_1+1\cdot \cfrac{1-(-1)^{n-1}}{1-(-1)}\\
&=2+\cfrac{1-(-1)^{n-1}}{2}=\cfrac{4+1-(-1)^{n-1}}{2}=\cfrac{5-(-1)^{n-1}}{2}
\end{align*}

Por lo que la solución a la recurrencia es:
\begin{align*}
f_n&=\cfrac{5+(-1)^n}{2}
\end{align*}

\subsubsection{Prueba}
Caso base:
\begin{align*}
f_1&=\cfrac{5-1}{2}=\cfrac{4}{2}=2.
\end{align*}

\begin{align*}
f_n&=f_{n-1}+(-1)^n\\
\cfrac{5+(-1)^n}{2}&=\cfrac{5+(-1)^{n-1}}{2}+(-1)^n\\
&=\cfrac{5+(-1)^{n-1}+2\cdot (-1)^n}{2}=\cfrac{5-(-1)^{n}+2\cdot (-1)^n}{2}\\
&=\cfrac{5+(-1)^n}{2}.
\end{align*}

\subsection{Ejempo 2}
Resolver la siguiente relación de recurrencia:
\begin{align*}
f_n&=-f_{n-1}+1,\,f_2=1,\,n>2.
\end{align*}

Dividimos por $(-1)^n$ de ambos lados.

\begin{align*}
\cfrac{f_n}{(-1)^n}&=\cfrac{-f_{n-1}}{(-1)^n}+\cfrac{1}{(-1)^n}\\
\cfrac{f_n}{(-1)^n}&=\cfrac{f_{n-1}}{(-1)^{n-1}}+(-1)^n\\
\end{align*}
Aplicamos el siguiente cambio de variable:
\begin{align*}
g_n&=\cfrac{f_n}{(-1)^n}
\end{align*}

Por lo que tenemos la siguiente relación de recurrencia:
\begin{align*}
g_n&=g_{n-1}+(-1)^n,\,g_2=\cfrac{f_2}{(-1)^2}=1,\,n>2
\end{align*}

Esta recurrencia es similar a la resuelta en en el primer ejemplo. Lo único que cambia es el índice y valor del caso base.

\subsection{Ejemplo 3}
Resolver la siguiente relación de recurrencia
\begin{align*}
nf_n&=(n+1)f_{n-1}+2n,\,n>0,\,f_0=0
\end{align*}

Primero reescribimos nuestra recurrencia.
\begin{align*}
f_n&=\cfrac{n+1}{n}f_{n-1}+2
\end{align*}

De aquí:
\begin{align*}
\cfrac{f_{n}}{n+1}&=\cfrac{f_{n-1}}{n}+\cfrac{2}{n+1}
\end{align*}

\subsection{Ejemplo 4}
$$f_n=-2\cdot n\cdot f_{n-1}+3\cdot n\cdot (n-1)\cdot f_{n-2},\,f_0=1,\,f_1=2,\,n>1 $$

Dividimos ambos lados sobre $n!$.
\begin{align*}
\cfrac{f_n}{n!}&=-2\cdot\cfrac{n\cdot f_{n-1}}{n!}+3\cdot \cfrac{n\cdot (n-1)\cdot f_{n-2}}{n!}
\end{align*}

Recordemos que:
$n!=n\cdot (n-1)!$
Entonces:
\begin{align*}
\cfrac{f_n}{n!}&=-2\cdot\cfrac{n\cdot f_{n-1}}{n\cdot (n-1)!}+3\cdot \cfrac{n\cdot (n-1)\cdot f_{n-2}}{n\cdot (n-1)\cdot (n-2)!}\\
\cfrac{f_n}{n!}&=-2\cdot \cfrac{f_{n-1}}{(n-1)!}+3\cdot \cfrac{f_{n-2}}{(n-2)!}
\end{align*}

Entonces definimos:
$$g_n=\cfrac{f_n}{n!}$$

Por lo que nos queda la siguiente recurrencia:
\begin{align*}
g_n&=-2\cdot g_{n-1}+3\cdot g_{n-2},\,g_0=1,\,g_1=2,\,n>1.
\end{align*}

\subsection{Ejercicios}
\begin{itemize}
\item $f_n=2^n\cdot f_{n-1}+1,\,f_1=2,\,n>1$
\item $n\cdot f_n=(n+t-1)\cdot f_{n-1},\,f_0=1,\,n>0$

\item $n\cdot f_n=3\cdot f_{n-1}+2^n,\,f_0=1,\,n>0$
\end{itemize}