\section*{Cadenas}

\subsection*{Ejemplo 1}

¿Cuantas subcadenas de longitud 4 pueden formarse con los caracteres de la cadena MISSISSIPPI?

\paragraph{Solución}

\begin{enumerate}
\item Primero definimos $n=11$. 

\item Posterior a esto, debemos definir el número de categorías, $c=4$.

\item Después de definir $n$ y $c$, determinamos la cantidad de categoría.
% --
\begin{align*}
n_1&=n_2=4; (I, S)\\
n_3&=2; (P)\\
n_4&=1; (M)
\end{align*}

\item Lo siguiente es escribir las particiones válidas para $m=4$. Este valor $m$ corresponde a la longitud de las cadenas que se nos pide contar.

Entonces:
\begin{table}[ht!]	\centering
	\begin{tabular}{c c c c}
		4 & \\
		3 & 1 \\
		2 & 2 \\
		2 & 1 & 1 \\
		1 & 1 & 1 & 1
	\end{tabular}
\end{table}
En este caso, todas las particiones son válidas.
\item Finalmente, procedemos a contar.
\begin{itemize}

\item Para \textbf{4}:
\textit{IIII}
\begin{align*}
n(\Se_1)&=\left[\binom{2}{1}\cdot\binom{4}{4}\right]
\end{align*}
Esto es: Hay dos categorías que tienen por lo menos 4 elementos. Escogemos una de esas categorías y ubicamos 4 elementos de esta. No es necesario permutar porque los 4 elementos son idénticos.

\item Para \textbf{3 1}: \textit{IIIM}

\begin{align*}
n(\Se_2)&=\underbset{3}{\left[\binom{2}{1}\cdot\binom{3}{3}\right]}\cdot\underbset{1}{\left[\binom{4-1}{1}\cdot\binom{1}{1}\right]}\cdot\underbset{\text{Permutar}}{\binom{4}{3,1}}
\end{align*}

\item Para \textbf{2 2}:
\textit{IIPP}
\begin{align*}
n(\Se_3)&=\left[\binom{3}{2}\binom{2}{2}\right]\cdot\binom{4}{2,2}
\end{align*}

\item Para \textbf{2 1 1}:
\textit{IIMP}
\begin{align*}
n(\Se_4)&=\underbset{\text{2}}{\left[\binom{3}{1}\binom{2}{2}\right]}\cdot\underbset{\text{1 1}}{\left[\binom{4-1}{2}\binom{1}{1}\right]}\cdot\binom{4}{2,1,1}
\end{align*}
\item Para \textbf{1 1 1 1}:
\textit{IMPS}
\begin{align*}
n(\Se_5)&=\left[\binom{4}{4}\binom{1}{1}\right]\cdot\binom{4}{1,1,1,1}
\end{align*}
\end{itemize}
\item Finalmente, sumamos todos estos elementos, y obtenemos:
\begin{align*}
n(\Se)&=n(\Se_1)+n(\Se_2)+n(\Se_3)+n(\Se_4)+n(\Se_5)
\end{align*}
\end{enumerate}

\subsection*{Ejemplo 2}
¿Cuántas cadenas de longitud 4 pueden formarse con los caracteres de la cadena PARALELEPIPEDO?

\begin{enumerate}
\item $n=14$.
\item $c=8$.
\item Eelementos de cada categoría: \begin{align*}
n_1&=n_2=3; (E, P)\\
n_3&=n_4=2; (A, L)\\
n_5&=n_6=n_7=n_8=1; (D, I, O,R)
\end{align*}
\item Particiones:
\begin{table}[!ht]
\centering
\begin{tabular}{c c c c }
3 & 1\\
2 & 2\\
2 & 1 & 1\\
1 & 1 & 1 & 1
\end{tabular}
\end{table}
\item 
\begin{itemize}
\item Para \textbf{3 1}:
\textit{EEEA}
\begin{align*}
n(\Se_{3,1})&=\left[\binom{2}{1}\cdot\binom{3}{3}\right]\cdot\left[\binom{8-1}{1}\cdot\binom{1}{1}\right]\cdot\binom{4}{3,1}
\end{align*}

\item Para \textbf{2 2}:
\textit{AAEE}
\begin{align*}
n(\Se_{2,2})&=\left[\binom{4}{2}\cdot\binom{2}{2}\right]\cdot\binom{4}{2,2}
\end{align*}
\item Para \textbf{2 1 1}:
\textit{AADE}
\begin{align*}
n(\Se_{2,1,1})&=\left[\binom{4}{1}\cdot\binom{2}{2}\right]\cdot\left[\binom{8-1}{2}\binom{1}{1}\right]\cdot\binom{4}{2,1,1}
\end{align*}
\item Para \textbf{1 1 1 1}:
\textit{ADEI}
\begin{align*}
n(\Se_{1,1,1,1})&=\left[\binom{8}{4}\cdot\binom{1}{1}\right]\cdot\binom{4}{1,1,1,1}
\end{align*}
\end{itemize}
\item Finalmente:
\begin{align*}
n(\Se)&=n(\Se_{3,1})+n(\Se_{2,2})+n(\Se_{2,1,1})+n(\Se_{1,1,1,1})
\end{align*}
\end{enumerate}

\subsection*{Ejemplo 3}
¿Cuántas cadenas de longitud 5 pueden formarse con los caracteres de la cadena PARALELEPIPEDO?

\begin{enumerate}
\item $n=14$.
\item $c=8$.
\item Eelementos de cada categoría: 
\begin{align*}
	n_1&=n_2=3; (E, P)\\
	n_3&=n_4=2; (A, L)\\
	n_5&=n_6=n_7=n_8=1; (D, I, O,R)
	\end{align*}
\item Particiones:
\begin{table}[ht!]
	\centering
	\begin{tabular}{c c c c c}
		3 & 2\\
		3 & 1 & 1\\
		2 & 2 & 1\\
		2 & 1 & 1 & 1\\
		1 & 1 & 1 & 1 & 1
	\end{tabular}
\end{table}
\item Conteo para cada partición
\begin{itemize}
\item Para \textbf{3 2}:
\begin{align*}
n(\Se_1)&=\left[\binom{2}{1}\cdot\binom{3}{3}\right]\cdot\left[\binom{4-1}{1}\cdot\binom{2}{2}\right]\cdot\binom{5}{3,2}
\end{align*}
\item Para \textbf{3 1 1}:
\begin{align*}
n(\Se_2)&=\left[\binom{2}{1}\cdot\binom{3}{3}\right]\cdot\left[\binom{8-1}{2}\cdot\binom{1}{1}\right]\cdot\binom{5}{3,1,1}
\end{align*}
\item Para \textbf{2 2 1}:
\begin{align*}
n(\Se_3)&=\underbset{2, 2}{\left[\binom{4}{2}\cdot\binom{2}{2}\right]}\cdot\underbset{1}{\left[\binom{8-2}{1}\cdot\binom{1}{1}\right]}\cdot\binom{5}{2,2,1}
\end{align*}
\item Para \textbf{2 1 1 1}:
\begin{align*}
n(\Se_4)&=\underbset{2, 2}{\left[\binom{4}{1}\cdot\binom{2}{2}\right]}\cdot\underbset{1}{\left[\binom{8-1}{3}\cdot\binom{1}{1}\right]}\cdot\binom{5}{2,1,1,1}
\end{align*}
\item Para \textbf{1 1 1 1 1}
\begin{align*}
n(\Se_5)&=\underbset{1,1,1,1,1}{\left[\binom{8}{5}\cdot\binom{1}{1}\right]}\cdot\binom{5}{1,1,1,1,1}
\end{align*}
\end{itemize}
\item Finalmente:
\begin{align*}
n(\Se)&=n(\Se_1)+n(\Se_2)+n(\Se_3)+n(\Se_4)+n(\Se_5)
\end{align*}
\end{enumerate}