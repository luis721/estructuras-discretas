\section{Funciones Generadoras Exponenciales (EGFs)}

Dada una secuencia $\seq{f_n}_{n\geq0}$, entonces su EGF $F(z)$ viene dada de la siguiente manera:

\begin{align*}
F(z)&=f_0\cdot\ek{0}+f_1\cdot\ek{1}+f_2\cdot\ek{2}+f_3\cdot\ek{3}+\cdots=\sum_{n\geq0}{f_n\cdot \ek{n}}.
\end{align*}

Esto es: Convertimos los términos de las secuencia el coeficientes de una serie de potencias.

\subsection{Ejemplo 1}

Dada la secuencia $\seq{1}_{n\geq0}$, hallar su función generadora.

\begin{align*}
F(z)&=1\cdot\ek{0}+1\cdot\ek{1}+1\cdot\ek{2}+1\cdot\ek{3}+\cdots=\sum_{n\geq0}{1\cdot \ek{n}}.
\end{align*}

Donde 
\begin{align*}
F(z)&=\sum_{n\geq0}{1\cdot\ek{n}}=e^z.
\end{align*}

Por lo tanto, la EGF de $\seq{1}_{n\geq0}$ es $e^z$.

Esto se escribe:
$$n![z^n]e^z=1.$$

\subsection{Ejemplo 2}
Dada la secuencia $\seq{2}_{n\geq0}$, hallar su función generadora.

\begin{align*}
F(z)&=2\cdot\ek{0}+2\cdot\ek{1}+2\cdot\ek{2}+2\cdot\ek{3}+\cdots=\sum_{n\geq0}{2\cdot \ek{n}}\\
&=2\cdot\left(1\cdot\ek{0}+1\cdot\ek{1}+1\cdot\ek{2}+1\cdot\ek{3}+\cdots\right)=2\cdot\left(\sum_{n\geq0}{1\cdot\ek{n}}\right)\\
F(z)&=2\cdot e^z.
\end{align*}

Entonces:
$$n![z^n]2\cdot e^z=2.$$

\subsection{Ejemplo 3}
Dada la secuencia $\seq{2}_{n\geq0}$, hallar su función generadora.
%
\begin{align*}
F(z)&=2\cdot\ek{0}+2\cdot\ek{1}+2\cdot\ek{2}+2\cdot\ek{3}+\cdots=\sum_{n\geq0}{2\cdot \ek{n}}\\
&=
\sum_{n\geq0}{(1+1)\cdot \ek{n}}=\underbset{e^z}{\sum_{n\geq0}{1\cdot \ek{n}}}+\underbset{e^z}{\sum_{n\geq0}{1\cdot \ek{n}}}\\
F(z)&=e^z+e^z=2\cdot e^z.
\end{align*}

Entonces:
$$n![z^n]2\cdot e^z=2.$$