\section{Cambio de Variable}

\subsection{Ejemplo 1: Transformación logarítmica}

\begin{align*}
f_n=\sqrt{f_{n-1}\cdot f_{n-2}}\,,f_0=1,\,f_1=1,\,n>1.
\end{align*}

En este caso aprovecharemos las siguientes propiedades del logaritmo:
\begin{align*}
\log(a\cdot b)=\log(a)+\log(b)
\end{align*}

además
\begin{align*}
\log(a^b)=b\log a
\end{align*}

Entonces, al aplicar logaritmo a ambos lados de la recurrencia:
\begin{align*}
\log f_n&=\cfrac{1}{2}\cdot \log (f_{n-1}\cdot f_{n-2})\\
&=\cfrac{1}{2}\cdot \left(\log f_{n-1}+\log{f_{n-2}}\right)
\end{align*}

Aquí aplicamos el siguiente cambio de variable:
\begin{align*}
g_n&=\log f_n,\,\underbset{\text{Casos base}}{{\color{magenta} g_0=\log f_0,\,g_1=\log f_1}}
\end{align*}
Reescribiendo la recurrencia a partir del cambio de variable se obtiene:
\begin{align*}
	g_n=\cfrac{1}{2}\cdot\left(g_{n-1}+g_{n-2}\right)
\end{align*}


\subsection{Ejemplo 2:}
\begin{align*}
f_n=\sqrt{1+f_{n-1}^2},\,f_0=0,\,n>0
\end{align*}

Si aplicamos logaritmo, obtenemos:
\begin{align*}
\log f_n=\cfrac{1}{2}\cdot \log\left(1+f_{n-1}^2\right)
\end{align*}

Lo que no es tan sencillo de resolver, porque no hay manera de deshacernos de la suma dentro del logaritmo.

Aplicaremos otra transformación al elevar ambos lados al cuadrado.
\begin{align*}
	f_n^2=1+f^2_{n-1}
\end{align*}

Procedemos a realizar el siguiente cambio de variable
$$g_n=f_{n}^2,\,g_0=f_0^2$$
Por lo tanto
\begin{align*}
g_n=g_{n-1}+1,\,g_0=0,\,n>0
\end{align*}

\subsubsection{Método de iteraciones}
\begin{align*}
k=1.\,g_n&=g_{n-1}+1\\
k=2.\,g_n&=g_{n-2}+1+1\\
k=3.\,g_n&=g_{n-3}+1+1+1\\
\end{align*}

Escribimos la ecuación paramétrica:
\begin{align*}
g(n,k)&=g_{n-k}+\sum_{i=0}^{k-1}{1}\\
&=g_{n-k}+k
\end{align*}
Lo siguiente es deshacernos del parámetro $k$. Esto se hace usando los casos base.
\begin{align*}
n-k=0\implies n=k\implies g_{n-k}=g_0
\end{align*}
Por lo tanto:
\begin{align*}
g_n&=g_0+n=0+n\\
g_n&=n.
\end{align*}

Una vez hemos hallado $g_n$, debemos hallar $f_n$.

Recordemos que
\begin{align*}
g_n&=f_{n}^2
\end{align*}
por lo tanto (luego de despejar $f_n$)
\begin{align*}
f_n=\sqrt{g_n}
\end{align*}
Por lo que finalmente obtenemos:
\begin{align*}
f_n=\sqrt{n}.
\end{align*}

\subsubsection{Una forma de demostrar}
A partir de la recurrencia original
\begin{align*}
	f_n=\sqrt{1+f_{n-1}^2}
\end{align*}
reemplazamos la expresión cerrada que se halló para $f_n$.

\begin{align*}
\sqrt{n}&=\sqrt{1+\left(\sqrt{n-1}\right)^2}\\
&=\sqrt{1+n-1}=\sqrt{n-0}\\
\sqrt{n}&=\sqrt{n}.
\end{align*}


\subsection{Ejemplo 3}
\begin{align*}
f_n&=f_{n-1}\cdot \sqrt{1-f_{n-2}^2},\,n>0,\,f_0=\frac{1}{2}
\end{align*}

Elevando al cuadrado ambos lados de la recurrencia se obtiene:
\begin{align*}
f_n^2&=f_{n-1}^2\cdot(1-f_{n-2}^2)=f_{n-1}^2-f_{n-1}^2\cdot f_{n-2}^2
\end{align*}

al intentar aplicar logaritmo sucede lo siguiente:
\begin{align*}
2\log f_n&=\log f_{n-1}^2+\log(1-f_{n-2}^2),
\end{align*}
ó
\begin{align*}
2\log f_n&=\log(f_{n-1}^2-f_{n-1}^2\cdot f_{n-2}^2),
\end{align*}
lo que muy probablemente no nos lleve a ningún lado.

\subsection{Ejemplo 4}
Resolver la siguiente relación de recurrencia:
\begin{align*}
f_n&=2\cdot f_{n-1}+2^n,\,n>0,\,f_0=1\\
\end{align*}

Dividimos $2^n$ de ambos lados.

\begin{align*}
\cfrac{f_n}{2^n}&=2\cdot\cfrac{f_{n-1}}{2^n} +\cfrac{2^n}{2^n}\\
\cfrac{f_n}{2^n}&=\cfrac{f_{n-1}}{2^{n-1}}+1
\end{align*}

Ahora hacemos el cambio de variable:
$$g_n=\cfrac{f_n}{2^n},\,g_0=\cfrac{f_0}{2^0}=1.$$

Por lo que tenemos la nueva recurrencia:
\begin{align*}
g_n&=g_{n-1}+1,\,n>0,\,g_0=1.
\end{align*}
Donde $g_n=n+1$.
Y a partir del cambio de variable que se hizo, al despejar $f_n$ se obtiene:

\begin{align*}
f_n&=2^n\cdot g_n=2^n\cdot (n+1).
\end{align*}
Por lo que tenemos la solución a nuestra recurrencia:
$$f_n=n2^n+2^n.$$

\subsubsection{Una forma de demostrar}
\begin{align*}
f_n&=2\cdot f_{n-1}+2^n\\
2^n\cdot (n+1)&=2\cdot 2^{n-1}\cdot n + 2^n\\
&=2^n\cdot n + 2^n\\
&=2^n\cdot (n+1).
\end{align*}